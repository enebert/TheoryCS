\documentclass[10pt, letterpaper]{article}
%Graphic and Diagram Setup
\usepackage[margin=1in]{geometry}
\usepackage{graphicx}
\usepackage{tikz}
\usepackage[all]{xy}

%Inserting Graphics
%An example of the use of this command is
% \begin{minipage}{4in}
%   \Fig{filename}
% \end{minipage}
\newcommand{\Fig}[1]{\includegraphics[width=0.5\textwidth]{#1}}

%Math Symbol Setup
\usepackage{ulem}
\usepackage{amsmath}
\usepackage{amsthm}
\usepackage{amssymb}

%Math Fonts
\usepackage{amsfonts}
\usepackage{mathrsfs}

%Absolute Value and Norm Notation
\newcommand{\abs}[1]{\left \lvert #1 \right \rvert}
\newcommand{\norm}[1]{\left \lVert #1 \right \rVert}

%Kernel of a map
\renewcommand{\ker}{\operatorname{Ker}}

%Lie Algebra of a Lie Group
\newcommand{\lie}[1]{\operatorname{Lie}(#1)}

%Fields and Blackboard Letters
\newcommand{\field}[1]{\mathbb{#1}}
\newcommand{\A}{\mathbb{A}}
\renewcommand{\P}{\mathbb{P}}
\newcommand{\N}{\mathbb{N}}
\newcommand{\Z}{\mathbb{Z}}

%Theorems and Definitions
\newtheorem{thm}{Theorem}
\newtheorem{lemma}{Lemma}
\newtheorem{cor}{Corollary}

\theoremstyle{remark}
\newtheorem{rem}{Remark}

\theoremstyle{definition}
\newtheorem{defn}{Definition}

%Fancy Header Setup
\usepackage{fancyhdr}
\fancyhf{}
\pagestyle{fancy}
\lhead{}
\chead{Universal Functions \& Undecidability}
\rhead{}
\lfoot{}
\cfoot{}
\rfoot{\thepage}
\renewcommand{\headrulewidth}{0.5pt}
\renewcommand{\footrulewidth}{0.1pt}

%Double Space
\usepackage{setspace}
\linespread{1.6}

%Document Content
\begin{document}
    \section*{Universal Functions \& Undecidability}

    \textbf{Goal: } Construct a set that is enumerable but not decidable.

    \begin{defn}
        A function $U:\N_0 \times \N_0 \rightarrow \N_0$ is said to be universal for th class of all computable functions of one 
        variable if:
        \begin{enumerate}
            \item [(a)] For any $n$, the function $U_n(x) = U(n,x)$ is computable. That is the sections of $U$ are computable.
            \item [(b)] all computable functions of one variable occur among the sections $U_n$.
        \end{enumerate}
    \end{defn}

    \begin{rem}
        Be reminded that it is possible that the $U$ and the $U_n$ are partially defined.
    \end{rem}

    \begin{thm}
        There exists a binary computable function universal for the class of unary computable functions.
    \end{thm}

    \begin{proof}
        Let's organize the class of computable unary functions into a computable sequence
        \[
            \{p_0, p_1, p_2, \ldots \}
        \]
        ascending by "length".

        We set $U(i,x)$ to be the output of the $i^{\text{th}}$ program run on the input $x$. Then the function $U$
        is just the desired computable function computed by the unary functions $p_i$.
    \end{proof}

    \begin{rem}
        A universal function can simulate any computable function on an input $x$.
    \end{rem}

    \begin{defn}
        A set $W \subseteq \N_0 \times \N_0$ is called a universal set for a certain class of sets of natural numbers, if all the sections 
        \[
            W_n = \{x \mid (n,x) \in W \}
        \]
        of the set $W$ belong to the class and there are no other sets in that class.
    \end{defn}

    \begin{thm}
        There exists an enumerable set in $\N_0 \times \N_0$ for the class of all enumerable sets of natural numbers.
    \end{thm}

    \begin{proof}
        Consider the domain of a universal function. It is a universal enumerable set as it satisfies one of the equivalent 
        definitions of enumerable. The sections $U_n$ are then also enumerable for any fixed $n$ by the same argument.
    \end{proof}

    \subsection*{Diagonal Constructon}

    \textbf{Q: } We can construct a computable universal function for the class of all computable functions in one variable.
    Can we do the same for the class of all total computable unary functions?

    \begin{thm}
        There is no total computable function of two variables universal to the class of all total computable functions 
        of one variable.
    \end{thm}

    \begin{proof}
        Let $U$ be an arbitrary total computable function of two variables. Consider the diagonal function $u(n) = U(n,n)$.
        The functions $u(n)$ and $d(n) = u(n)+1$ differ. The function $d(n)$ differs from all of the sections $U_n$. We conclude
        that the function $U$ is not universal.
    \end{proof}

    \begin{rem}
        The difference between have a general computable function and a total function? A general computable function is 
        potentially partially defined. It is possible that $d(n) = U(n,n) + 1$ and $u(n)$ have the same "value when undefined.
    \end{rem}

    \begin{thm}
        There exists a computable function $d$ such that no computable function $f:\N_0 \rightarrow \N_0$ can differ from it 
        everywhere.
    \end{thm}

    \begin{rem}
        There exists some $n \in \N_0$ such that either $f(n) = d(n)$ or both are undefined.
    \end{rem}

    \begin{proof}
        Let $U$ be a universal function for the class of computable functions. Any computable function $f$ coincides 
        with $U_n$ for a certain $n$. So we should take $d(n)$ to be $f(n) = U_n(n) = U(n,n)$.
    \end{proof}

    \begin{thm}
        There exists a computable function that has no total computable extension.
    \end{thm}

    \begin{proof}
        From the previous theorem, we defined $d(n) = U_n(n)$ and saw that it is a computable function that shares a value 
        with any other computable function. Here we define $\hat{d}(n) = d(n) + 1$. Let $\bar{d}$ be some total extension of $\hat{d}$.
        Note the following:
        \begin{align*}
            d(n) \text{ defined } & \Rightarrow \hat{d}(n) = d(n)+1 \text{ defined } \\
                                  & = \bar{d}(n) \\
                                  & \neq d(n)
        \end{align*}
        and if $d(n)$ is undefined then $d(n) \neq \bar{d}(n)$ because $\bar{d}$ is total. We conclude that $\bar{d}$ differs 
        from $d$ everywhere. From the previous theorem $\bar{d}$ cannot be computable.
    \end{proof}

    \subsection*{Enumerable Undecidable Set}

    We fulfill our goal of finding an enumerable undecidable set. recall that we need to obtain an enumerable set 
    with a complement that is not enumerable.

    \begin{thm}
        There exists an enumerable undecidable set in $\N_0$.
    \end{thm}

    \begin{proof}
        Let $f:\N_0 \rightarrow \N_0$ be a computable function that has no total computable extension. Since $f$ is 
        computable we know that $\operatorname{Dom} f$ is enumerable. If $\operatorname{Dom} f$ were decidable then the 
        characteristic function would be computable and therefore
        g(x) = \[
            \begin{cases}
                f(x) & \text{ if } x \in \operatorname{Dom} f \\
                0 & \text{ if } x \notin \operatorname{Dom} f \\
            \end{cases}
        \]
        would be computable which is a total extension of $f$. This is a contradicition to our assumption that $f$ has no 
        total computable extension. We conclude that $\operatorname{Dom} f$ is undecidable.
    \end{proof}

    \begin{cor}[Halting Problem]
        The decision problem: For a given algorithm $M$ decide if $M$ halts on $x$ is undecidable.
    \end{cor}

    \begin{rem}
        this is a rephrasing of: The domain of the function $U$ is also an enumerable undecidable st of pairs.
    \end{rem}

    \subsection*{Enumerable Inseperable Sets}

    \begin{thm}
        There exists a computable function that takes only the values of 0 and 1 and has no total computable extension.
    \end{thm}

    \begin{proof}
        Consider the function 
        \[
            d'(x) = \begin{cases}
                1 &\text{ if } d(x) = 0 \\
                0 &\text{ if } d(x) > 0 \\
                undef &\text{ if } d(x) \text{ is undefined} \\
            \end{cases}
        \]

        where $d(x)$ is a computable function that shares a value with any other computable function. Any total 
        extension of $d'$ will differ from $d$ everywhere and therefore not computable. (?)
    \end{proof}

    \subsection*{Simple Sets: The Post Construction}

    \textbf{Goal: } E. Post's constructon of an enumerable undecidable set. 

    \begin{defn}
        \begin{enumerate}
            \item A set is said to be immune if it is infinite but does not contain infinitely many enumerable subsets.
            \item An enumerable set is said to be simple if its complement is immune.
        \end{enumerate}
    \end{defn}

    \begin{defn}
        We say that a set $C$ separates two disjoint sets $X$ and $Y$ if $C$ contains one of them and has empty intersection 
        with the other.
    \end{defn}

    \begin{thm}
        There exist two disjoint enumerable sets $X$ and $Y$ that cannot be separated by any decidable set.
    \end{thm}

    \begin{proof}
        Let $d$ be a computable function that only takes the values 0 and 1 with no total computable extension. Set 
        \begin{align*}
            X &= \{x \mid d(x) = 1\}
            Y &= \{x \mid d(x) = 0\}
        \end{align*}

        It is clear that both $X$ and $Y$ are enumerable. Suppose that they can be separated by a decidable set $C$.
        WLOG suppose $X \subseteq C$ and $C \cap Y = \varnothing$. Note that the characteristic function of $C$ extends $d$ which 
        contradicts our assumption that $d$ has not total computable extension.
    \end{proof}

    \begin{thm}
        There exists a simple set.
    \end{thm}

    We want to find / construct an enumerable set $S$ with an immune complement. If such a set $S$ exists then it must share 
    a common point with any infinite enumerable subset. To guarantee this, we will add to $S$ an element of each infinite enumerable 
    set $V$. We also need to ensure that infinitely many elements are left outside of $S$. We will add to $S$ only sufficiently 
    large elements to make this happen.

    \textbf{Construction: } Let $W$ be a universal enumerable set of pairs. It's sections $W_i$ are our enumerable sets in 
    $\N_0$. We refer to $W_i$ as the $i^{\text{th}}$ enumerable set. Define 
    \[
        T = \{(i,x) \mid x \in W_i \text{ and } x > 2i\}
    \]

    The set $T$ is enumerable as 
    \[
        W_i \cap \{x > 2i \}
    \]
    is the intersection of an enumerable set and a decidable set. Let $M$ enumerate the elements of $T$ by omitting the 
    pairs where the first element has already occurred earlier. Then a certain enumerable subset $T'$ of $T$ remains. Now consider 
    the enumerable set $S$ of the second elements of all pairs contained in $T'$.

    \textbf{Claim: } The set $S$ has nonempty intersection with any infinite enumerable set. 

    Any infinite section $W_i$ contains elements larger that $2i$. The set $T$ contains at least one element with first 
    component $i$. The second component of this pair belongs both to $S$ and to $W_i$. On the other hand, the set $S$ has 
    infinite complement since at most $n+1$ different numbers from 0 to $2n$ belong to $S$.
\end{document}
