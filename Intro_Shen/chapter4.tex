\documentclass[10pt, letterpaper]{article}
%Graphic and Diagram Setup
\usepackage[margin=1in]{geometry}
\usepackage{graphicx}
\usepackage{tikz}
\usepackage[all]{xy}

%Inserting Graphics
%An example of the use of this command is
% \begin{minipage}{4in}
%   \Fig{filename}
% \end{minipage}
\newcommand{\Fig}[1]{\includegraphics[width=0.5\textwidth]{#1}}

%Math Symbol Setup
\usepackage{ulem}
\usepackage{amsmath}
\usepackage{amsthm}
\usepackage{amssymb}

%Math Fonts
\usepackage{amsfonts}
\usepackage{mathrsfs}

%Absolute Value and Norm Notation
\newcommand{\abs}[1]{\left \lvert #1 \right \rvert}
\newcommand{\norm}[1]{\left \lVert #1 \right \rVert}

%Kernel of a map
\renewcommand{\ker}{\operatorname{Ker}}

%Lie Algebra of a Lie Group
\newcommand{\lie}[1]{\operatorname{Lie}(#1)}

%Fields and Blackboard Letters
\newcommand{\field}[1]{\mathbb{#1}}
\newcommand{\A}{\mathbb{A}}
\renewcommand{\P}{\mathbb{P}}
\newcommand{\N}{\mathbb{N}}
\newcommand{\Z}{\mathbb{Z}}

%Theorems and Definitions
\newtheorem{thm}{Theorem}
\newtheorem{lemma}{Lemma}
\newtheorem{cor}{Corollary}

\theoremstyle{remark}
\newtheorem{rem}{Remark}

\theoremstyle{definition}
\newtheorem{defn}{Definition}

%Fancy Header Setup
\usepackage{fancyhdr}
\fancyhf{}
\pagestyle{fancy}
\lhead{}
\chead{\bfseries Properties of G\"{o}del Numberings}
\rhead{}
\lfoot{}
\cfoot{}
\rfoot{\thepage}
\renewcommand{\headrulewidth}{0.5pt}
\renewcommand{\footrulewidth}{0.1pt}

%Double Space
\usepackage{setspace}
\linespread{1.6}

%Document Content
\begin{document}
\subsection*{Guiding Questions}

\begin{enumerate}
    \item[(a)] Is the set of numbers of the empty function in a certain G\"{o}del numbering decidable?
    \item[(b)] Is it possible to determine whether a function is empty given its number in a G\"{o}del numbering? 
\end{enumerate}

\begin{rem}
    Any two G\"{o}del numberings can be reduced to each other. Given a number of a function in one G\"{o}del numbering it is possible 
    to obtai a number of the same function in the other numbering. Conclusion? The answer to the above quesiton is independent 
    of the G\"{o}del numbering considered.
\end{rem}

\begin{thm}
    Let $U$ be an arbitrary G\"{o}del universal function. Then the set of all the numbers $n$ such that the functions $U_n$ is 
    empty is undecidable. 
\end{thm}

\begin{proof}
    We use the method of reduction.

    \textbf{Claim:} If this set were decidable, then any enumerable set would be decidable.

    Let $K$ be an arbitarary enumerable undecidable set. Consider the following binary computable function:
    \[
        V(n,x) = \begin{cases}
            0 &\text{ if } n \in K \\
            \text{undefined} &\text{ if } n \notin K \\
        \end{cases}
    \]
    Essentially this is the semicharacteristic function for $K$. The function $V$ has the sections:
    \begin{enumerate}
        \item For $n \in K$, $V_n \equiv 0$.
        \item For $n \notin K$, $V_n$ is nowhere defined and is therefore the empty function.
    \end{enumerate}
    Since $U$ is a G\"{o}del universal function, there exists a total computable function $s$ such that 
    \[
        V(n,x) = U(s(n), x) \hspace*{.15in} \forall n,x
    \]
    that is $V_n = U_{s(n)}$. Hence, the value $s(n)$ is a number for the zero function where $n \notin K$. If the set of $U$-numbers 
    for the empty function were decidable, we could apply that algorithm to $s(n)$ and $K$ would be decidable. One of the things we 
    can observed now: Any finite set is decidable so the cardinality of the numbers in any G\"{o}del numbering is infinite.

    \textbf{Claim: } The set of numbers of the empty function is not enumerable either.

    The set of numbers for any numbering of nonempty functions is enumerable. How? If $U(n,x)$ is defined for some $x$ then print $n$. We saw 
    that Post's theorem said if the complement of an undecidable set is enumerable then the set itself is nonenumerable.
\end{proof}

\begin{thm}[Rice-Uspensky]
    Let $\mathscr{F}$ denote the class of all unary computable functions. Let $\mathscr{A} \subset \mathscr{F}$ be an arbitrary 
    nontrivial property of computable functions with $\mathscr{A} \neq \varnothing$ and the inclusion is proper. Let $U$ be a G\"{o}del universal 
    function. Then it is impossible to determine algorithmically wheter a computable function with a given $U$-number has the property $\mathscr{A}$.

    \framebox{
        The set $\{n \mid U_n \in \mathscr{A}\}$ is undecidable.
    }
\end{thm}

\begin{proof}
    WLOG we may assume that the empty function $\zeta$ belongs to $\mathscr{A}$. If it doesn't replace $\mathscr{A}$ with its complement. Let 
    $\xi \in \mathscr{F} \setminus \mathscr{A}$ be arbitarary. We repeat the previous proof with 
    \[
        V(n,x) = \begin{cases}
            \xi(x) &\text{ if } n \in K \\
            \text{undefined} &\text{ if } n \notin K \\
        \end{cases}
    \]
    Again the function $V$ is computable as an adjustment of the semicharacteristic function on $K$. For $n \in K$ then $V_n$ conincides with 
    $\xi$ and $n \notin K$ it conincides with $\zeta$. Hence, $V_n \in \mathscr{A}$ if and only if $n \notin K$.

    If the set $\{n \mid U_n \in \mathscr{A}\}$ were decidable then $V_n \in \mathscr{A}$ is algorithmically decidable for a given $n$. Hence, we can decide 
    if $n \in K$ or not which is contradictory to our assumption that $K$ is undecidable.

    \textbf{Claim: } If it were possible to recognize the property by $U$-numbers, then any two disjoint enumerable sets $P$ and $Q$ could be separated by 
    a decidable set.

    Choose any two functions $\xi$ and $\eta$ and consider the function:
    \[
        V(n,x) = \begin{cases}
            \xi(x) &\text{ if } n \in P \\
            \eta(x) &\text{ if } n \in Q \\
            \text{undefined} &\text{ if } n \notin P \cup Q 
        \end{cases}
    \]
    Note that for any $n$ and $x$ we can wait for $n$ to appear in $P$ or $Q$ and then compute $\xi(x)$ or $\eta(x)$ as needed. So 
    $V$ is computable. If $n \in P$ then $V_n \equiv \xi$, if $n \in Q$ then $V_n \equiv \eta$. By verifying if $V_n \in \mathscr{A}$ we 
    could decidably separate $P$ from $Q$ which is a contradiction.
\end{proof}

\subsection*{New Numbers of Old Functions}

We've seen that the set of numbers of any specific funtion in a a G\"{o}del numbering is undecidable and therefore infinite. We now prove a 
stronger statement.

\begin{thm}
    Let $U$ be a universal G\"{o}del function. Then there exists a total binary function $g$ such that for any $i$ the values 
    $g(i,0), g(i,1), \ldots$ are different $U$-numbers of the function $U_i$.
\end{thm}

\begin{proof}
    Let $h$ be an arbitrary function. We show that there exists an algorithm that finds infinitely many different $U$-numbers of the function $h$.
    Let $P$ be an enumerable undecidable set. Consider a computable function 
    \[
        V(n,x) = \begin{cases}
            h(x) &\text{ if } n \in P \\
            \text{undefined} &\text{ if } n \notin P 
        \end{cases}
    \]
    There are two functions in play for the sections, namely $h$ and $\zeta$.

    \textbf{Case 1:} $h \neq \zeta$

    Since $U$ is a G\"{o}del universal function, there exists a converter $s$ that transforms $V$-numbers into $U$-numbers. That is 
    \begin{itemize}
        \item $n \in P \Rightarrow U_{s(n)} = V_n = h$
        \item $n \notin P \Rightarrow U_{s(n)} = V_n = \zeta$
    \end{itemize}
    It follows that if $p(0), p(1), \ldots$ is a computable enumeration of the set $P$, then all of the $U$-numbers $s(p(0)), s(p(1)), \ldots$ 
    are numbers of the function $h$.

    \textbf{Claim: } The set $\{s(p(0)), s(p(1)), \ldots\}$ is infinite.

    Suppose that this is not the case, and the set $X =\{s(n) \mid n \in P\}$ is finite. Then $X$ is decidable and 
    \begin{itemize}
        \item $n \in P \Rightarrow s(n) \in X$
        \item $n \notin P$ is a number of $\zeta$ and does not belong to $X$.
    \end{itemize}
    One now sees that $n \in P$ if and only if $s(n) \in X$ and the decidability of $X$ impllies the decidability of $P$ which is a 
    contradiction.

    \textbf{Case 2: } Suppose $h = \zeta$.

    Let $\xi$ be a computable function that is defined at least one point. Consider two enumerable inseparable sets $P$ and $Q$ and the computable 
    function 
    \[
        V(n,x) = \begin{cases}
            h(x) &\text{ if } n \in P \\
            \xi(x) &\text{ if } n \in Q \\
            \text{undefined} &\text{ if } n \notin P \cup Q
        \end{cases}
    \]

    Let $s$ be a converter from $V$-numbers to $U$-numbers. Then 
    \begin{itemize}
        \item $n \in P \Rightarrow U_{s(n)} = h$
        \item $n \in Q \Rightarrow U_{s(n)} = \xi$
        \item $n \notin P \cup Q \Rightarrow U_{s(n)} = \zeta$
    \end{itemize}
    As before $s(p(0)), s(p(1)), \ldots$ are numbers of the function $h$.

    \textbf{Claim: } If $h \neq \xi$ then the set $X$ of the numbers $h, s(p(0)), s(p(1)), \ldots$ is undecidable.

    If $X$ were decidable, then we could separate $P$ from $Q$ by namely $P \subset \{n \mid s(n) \in X\}$ and $Q \cap X = \varnothing$.
    We conclude that we can produce numbers of $h$. If we don't know ehther $h = \zeta$ or not we may run these two 
    constructions in parallel until one of them produces a new number.

    We now consider the following computable functions:

    \begin{align*}
        V_1(\sigma(i,n),x) & = \begin{cases}
            U(i,x) &\text{ if } n \in P \\
            \text{undefined} &\text{ otherwise}
        \end{cases} \\
        V_2(\sigma(i,n),x) &= \begin{cases}
            U(i,x) &\text{ if } n \in P \\
            0 &\text{ if } n \in Q \\
            \text{ undefined } &\text{ if } n \notin P \cup Q
        \end{cases}
    \end{align*}

    Since $U$ is a universal G\"{o}del function, we can find computable total functions $s_1$ and $s_2$ such that 
    \begin{align*}
        V_1(\sigma(i,n),x) &= U(s_1(\sigma(i,n)),x)
        V_2(\sigma(i,n),x) &= U(s_2(\sigma(i,x)),x)
    \end{align*}
    Let $p$ be a total unary function such that $p = \{p(0), p(1), \ldots\}$. The desired function $g$ is obtained as follows:
    $g(i,k)$ is the $k^{\text{th}}$ number is the sequence $s_1(\sigma(i,p(0))), s_2(\sigma(i,p(0))), \ldots$.
\end{proof}

\subsection*{Isomorphism of G\"{o}del Numbers}

\begin{thm}[Rogers]
    Let $U_1$ and $U_2$ be two G\"{o}del universal funtions for the class of unary computable functions. Then there exist two
    total mutually inverse computable function $s_{12}$ and $s_{21}$ such that 
    \begin{align*}
        U_1(n,x) &= U_2(s_{12}(n),x) \\
        U_2(n,x) &= U_1(s_{21}(n),x) \\
    \end{align*}
    for any $n,x$.
\end{thm}

\begin{rem}
    We're saying that any two G\"{o}del numberings differ from each other only by a permutation of the numbers.
\end{rem}
\end{document}
