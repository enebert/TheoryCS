\documentclass[10pt, letterpaper]{article}
%Graphic and Diagram Setup
\usepackage[margin=1in]{geometry}
\usepackage{graphicx}
\usepackage{tikz}
\usepackage[all]{xy}

%Inserting Graphics
%An example of the use of this command is
% \begin{minipage}{4in}
%   \Fig{filename}
% \end{minipage}
\newcommand{\Fig}[1]{\includegraphics[width=0.5\textwidth]{#1}}

%Math Symbol Setup
\usepackage{ulem}
\usepackage{amsmath}
\usepackage{amsthm}
\usepackage{amssymb}

%Math Fonts
\usepackage{amsfonts}
\usepackage{mathrsfs}

%Absolute Value and Norm Notation
\newcommand{\abs}[1]{\left \lvert #1 \right \rvert}
\newcommand{\norm}[1]{\left \lVert #1 \right \rVert}

%Kernel of a map
\renewcommand{\ker}{\operatorname{Ker}}

%Lie Algebra of a Lie Group
\newcommand{\lie}[1]{\operatorname{Lie}(#1)}

%Fields and Blackboard Letters
\newcommand{\field}[1]{\mathbb{#1}}
\newcommand{\A}{\mathbb{A}}
\renewcommand{\P}{\mathbb{P}}
\newcommand{\N}{\mathbb{N}}
\newcommand{\Z}{\mathbb{Z}}

%Theorems and Definitions
\newtheorem{thm}{Theorem}
\newtheorem{lemma}{Lemma}
\newtheorem{cor}{Corollary}

\theoremstyle{remark}
\newtheorem{rem}{Remark}

\theoremstyle{definition}
\newtheorem{defn}{Definition}

%Fancy Header Setup
\usepackage{fancyhdr}
\fancyhf{}
\pagestyle{fancy}
\lhead{}
\chead{\bfseries Oracle Computations}
\rhead{}
\lfoot{}
\cfoot{}
\rfoot{\thepage}
\renewcommand{\headrulewidth}{0.5pt}
\renewcommand{\footrulewidth}{0.1pt}

%Double Space
\usepackage{setspace}
\linespread{1.6}

%Document Content
\begin{document}
    If a set $B$ is $m$-reducible to a decidable set $A$, $B \leq_m A$, then $B$ is also decidable. Suppose $A$ is undecidable, but we have 
    access to an oracle for $A$ that answers questions about the membership of numbers in $A$. In which case, we can use it to answer 
    questions about $B$. Here's a general scheme:

    \begin{enumerate}
        \item [(a)] Only one question is asked.
        \item [(b)] The answer is taken as the answer to the original question.
    \end{enumerate}

    Here are a couple of instances that are related but don't quite follow the rules above:

    \begin{enumerate}
        \item [(a)] Given an oracle for the set $A$ can we answer questions about $A^c$? Yes, but (b) in the scheme above is being reversed.
        \item [(b)] Given an oracle for $A$ can we answer questions about $A \times A$? Yes, but it seems like (a) in the scheme above isn't quite 
        right as we're asking two questions.
    \end{enumerate}

    We now seek a more general notion of reducibility:

    \begin{defn}
        \begin{enumerate}
            \item [(a)] Let us say that $B$ is Turing Reducible to $A$ if there exists an algorithm that decides the set $B$ using an oracle answering 
            questions about $A$. We denote this by $B \leq_T A$.
            \item [(b)] If $B$ is reducible to $A$, we also say that $B$ is $A$-decidable.
        \end{enumerate}
    \end{defn}

    \begin{thm}
        \begin{enumerate}
            \item [(a)] If $B \leq_m A$, then $B \leq_T A$.
            \item [(b)] $A \leq_T \N_0 \setminus A$ for any $A$.
            \item [(c)] If $A \leq_T B$ and $B \leq_T C$ then $A \leq_T C$.
            \item [(d)] If $A \leq_T B$ and $B$ is decidable, then $A$ is decidable.
        \end{enumerate}
    \end{thm}

    \begin{rem}
        Notice that (b) implies that it's possible that a nonenumerable set can be Turing Reducible to an enumerable set, which is 
        impossible for $m$-reducibility.
    \end{rem}

    We can define the notion of an $A$-computable funciton similarly to how e defined what it meant to be $A$-decidable.

    \begin{defn}
        A funcction $f$ is said to be $A$-computable if there exists an algorithm $M$, with calls to an oracle, that computes $f$ if the calls are correctly 
        answered by the $A$-oracle.
    \end{defn}

    \begin{itemize}
        \item If $f$ is defined at $x$, then $f(x)$ is returned and the algorithm halts.
        \item If $x \notin \operatorname{Dom} f$, then it does not halt.
    \end{itemize}

    \begin{thm}
        A partial function $f$ is computable relative to a total function $\alpha$ if and only if it is computable relative to the graph of the 
        function $\alpha$.
    \end{thm}

    We now work through a whole slew of definitions:

    \begin{defn} \leavevmode
        \begin{enumerate}
            \item [(a)] A function with natural values defined on a finite set of $\N_0$ is called a pattern. Any patter is defined by a set of 
            pairs $(\text{ arg }, \text{ value })$.
            \item [(b)] Two patterns are called coherent if the union of their graphs is the graph of a function. There is no place where both are 
            defined and take on different values.
            \item [(c)] Let $(x,y,t)$ be a triple where $x,y \in \N_0$ and $t$ is a pattern. We will say that two triples $(x_1, y_1, t_1)$ and $(x_2, y_2, t_2)$ are 
            incompatible if the patterns are coherent if $x_1=x_2$ but $y_1 \neq y_2$.
            \item [(d)] A set $M$ will be said to be consistent if it contains no incompatible triples.
        \end{enumerate}
    \end{defn}

    We now have the following construction:

    Let $M$ be a consistent set and $\alpha$ some function. Consider triples $(x,y,t) \in M$ such that $\Gamma(t) \subset \Gamma(\alpha)$.
    All patters are coherent so $M$ is consistent. No two of the triples can have equal first and different second components. If we omit the 
    $3^{\text{rd}}$ element of the triples, we get the graph of a function (which is probably partial). This function will be 
    denoted $M[\alpha]$.

    \begin{thm}
        A partial function $\N_0 \rightarrow \N_0$ is computable relative to a total function $\alpha: \N_0 \rightarrow \N_0$ if and only if 
        there exists an enumerable consistent set of triples $M$ such that $f = M[\alpha]$.
    \end{thm}
    \subsection*{Relativization}

    Let us fix a total function $\alpha$ the entire theory of computable functions can be relativized with respect to $\alpha$. So the theory 
    we've developed can be applied to $\alpha$-computable functions. The proofs are the same with small changes. In particular, we can 
    defined the notion of a set enuemrable with respect to $\alpha$ or $\alpha$-enumerable:
    \begin{enumerate}
        \item [(a)] As the domain of an $\alpha$-computable function.
        \item [(b)] As the range of an $\alpha$-computable function.
        \item [(c)] As the projection of an $\alpha$-decidable set.
    \end{enumerate}

    Let $E$ be an arbitrary set of pairs $(x,t)$ where $x$ is a numer and $t$ is a pattern. We take a total function $\alpha$ and pick out from 
    the set $E$ the pairs whose $2^{\text{nd}}$ components are parts of $\alpha$. The first components will be denoted by $E[\alpha]$.

    \begin{thm}
        A set $X$ is $\alpha$-enumerable if and only if $X=E[\alpha]$ for some enumerable set $E$.
    \end{thm}

    \begin{proof}
        Let $X$ be the domain of a function $f$ that is $\alpha$-computable. Let $f=M[\alpha]$ for some enumerable set $M$ that is 
        consistent. Let us delete the $3^{\text{rd}}$ component from each triple in $M$. We obtain an enumerable set of pairs, call it $E$.
        Then $E[\alpha]$ is the domain of $f$ so $E[\alpha]=X$.

        Conversely, suppose that $X=E[\alpha]$ for some function $\alpha$. Now consider the set $M$ obtained by inserting 0 between the 
        components for each pair of $E$. The set $M$ is consistent and $M[\alpha] \equiv 0$ defined on $X=E[\alpha]$.
    \end{proof}

    \begin{thm}
        Let $\alpha$ be a total function. There exists a binary computable function universal for the class of unary $\alpha$-computable functions.
    \end{thm}
\end{document}
