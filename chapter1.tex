\documentclass[10pt, letterpaper]{article}
%Graphic and Diagram Setup
\usepackage[margin=1in]{geometry}
\usepackage{graphicx}
\usepackage{tikz}
\usepackage[all]{xy}

%Inserting Graphics
%An example of the use of this command is
% \begin{minipage}{4in}
%   \Fig{filename}
% \end{minipage}
\newcommand{\Fig}[1]{\includegraphics[width=0.5\textwidth]{#1}}

%Math Symbol Setup
\usepackage{ulem}
\usepackage{amsmath}
\usepackage{amsthm}
\usepackage{amssymb}

%Math Fonts
\usepackage{amsfonts}
\usepackage{mathrsfs}

%Absolute Value and Norm Notation
\newcommand{\abs}[1]{\left \lvert #1 \right \rvert}
\newcommand{\norm}[1]{\left \lVert #1 \right \rVert}

%Kernel of a map
\renewcommand{\ker}{\operatorname{Ker}}

%Lie Algebra of a Lie Group
\newcommand{\lie}[1]{\operatorname{Lie}(#1)}

%Fields and Blackboard Letters
\newcommand{\field}[1]{\mathbb{#1}}
\newcommand{\A}{\mathbb{A}}
\renewcommand{\P}{\mathbb{P}}
\newcommand{\N}{\mathbb{N}}
\newcommand{\Z}{\mathbb{Z}}

%Theorems and Definitions
\newtheorem{prop}{Proposition}
\newtheorem{thm}{Theorem}
\newtheorem{lemma}{Lemma}
\newtheorem{cor}{Corollary}

\theoremstyle{remark}
\newtheorem{rem}{Remark}

\theoremstyle{definition}
\newtheorem{defn}{Definition}

%Fancy Header Setup
\usepackage{fancyhdr}
\fancyhf{}
\pagestyle{fancy}
\lhead{}
\chead{\bfseries Computability}
\rhead{}
\lfoot{}
\cfoot{}
\rfoot{\thepage}
\renewcommand{\headrulewidth}{0.5pt}
\renewcommand{\footrulewidth}{0.1pt}

%Double Space
\usepackage{setspace}
\linespread{1.6}

%Document Content
\begin{document}

Let's begin with a definition:

\begin{defn}
	A function $f: \N_0 \rightarrow \N_0$ is said to be computable if there exists an algorithm $M$ that computes $f$. The algorithm $M$:
	\begin{enumerate}
		\item[(a)]  halts on the input $n$ and outputs $f(n)$ when $f$ is defined for an $n \in \N_0$.
		\item[(b)] does not halt on the input $n$ whenever $f$ is undefined for $n \in \N_0$.
	\end{enumerate}
\end{defn}

\begin{proof}(Not Really) \\
Let $f: \N_0 \rightarrow \N_0$ be a partial computable function. Extend $f$ to a total function
\[
	f(n) = \begin{cases}
		f(n) &\text{ if } n \in \operatorname{Dom} f \\
		0 & \text { otherwise}\\
	\end{cases}
\]

The algorithm $M$ does not halt on $n \notin \operatorname{Dom} f$. However, $M$ is the only things we have to decide if $f(n)$ is defined. How do we know when to halt $M$ and output $g(n) = 0$?
\end{proof}

\subsection*{Decidable Sets}

\begin{defn}
	A set $X \subseteq \N_0$ is said to be decidable if there exists an algorithm $M$ that determines if $n \in \N_0$ belongs to $X$
\end{defn}

\noindent \textbf{Q: } What does this mean? \\

\noindent \textbf{A: } The characteristic function for $X$ 
\[
	\chi_X(n) = \begin{cases}
		1 & \text{ if } n \in X \\
		0 & \text{ otherwise } \\
	\end{cases}
\]
is computable.

\begin{rem}
	\item[(a)] Any finite set is decidable. We can make an algorithm that outputs 1 on a finite number of elements.
	\item[(b)] The collection of decidable sets is closed under intersection, union, and set difference.
\end{rem}

\subsection*{Enumerable Sets}

Any set that satisfies any of the following equivalent conditions is said to be \emph{enumerable}.

\begin{prop}
	Let $X \subseteq \N_0$. TFAE:
	\begin{enumerate}
		\item[(a)] There exists an algorithm $M$ that outputs the elements of $X$ and only those elements.
		\item[(b)] $X$ is the domain of a computable function.
		\item[(c)] $X$ is the range of a computable function.
		\item[(d)] The semicharacteristic function for $X$
		\[
			f_X(n) = \begin{cases}
				0 &\text{ if } n \in X \\
				\text{ undefined } &\text{ otherwise } \\
			\end{cases}
		\]
	\end{enumerate}
\end{prop}

\begin{proof}
	(a) $\Rightarrow$ (b,d) Suppose that $X$ is enumerated by an algorithm $M$. Then we can compute the semicharacteristic function for $X$:
		\begin{enumerate}
			\item[(i)] Execute $M$ until $n$ is printed.
			\item[(ii)] Output 0 and terminate.
		\end{enumerate}
		
	(c) $\Rightarrow$ (b) Suppose $X$ is in the domain of a computable function $f$ and $N$ is the algorithm that computes $f$. We'll run $N$ in parallel on 0, 1, 2, $\ldots$ gradually increasing the number of involved inputs:
		\begin{enumerate}
			\item[(i)] Run one step of $N$ on 0 and 1.
			\item[(ii)] Run two steps of $N$ on 0, 1, and 2.
			\item[(iii)] Run three steps of $N$ on each input 0, 1, 2, and 3.
		\end{enumerate}
		All arguments on which $N$ terminates are printed out as soon as they are detected. We avoid having $N$ go into an infinite loop on an undefined element $n \in \N_0$ by only performing a finite number of steps from the algorithm $N$ each time. We can modify this algorithm to print the results returned rather than the arguments where it terminates to prove the converse (b) $\Rightarrow$ (c).

(b) $\Rightarrow$ (c) We've seen that $X$ is the domain of a computable function. This function is computed by an algorithm $N$. Define
\[
	b(x) = \begin{cases}
		x &\text{ if } M \text{ terminates on } X \\
		\text{ undefine } & \text{ otherwise } \\
	\end{cases}
\]
\end{proof}

\begin{defn}
	A set $X \subseteq \N_0$ is enumerable if either:
	\begin{enumerate}
		\item[(i)] $X = \varnothing$
		\item[(ii)] $X$ is in the range of a total computable function.
	\end{enumerate}
\end{defn}

This definition turns out to be equivalent to all of the others. Suppose $X$ is enumerated by an algorithm $M$. Let $x_0 \in X$ be arbitrary. Define the total function
\[
	a(n) = \begin{cases}
		t & \text{ if } M \text{ returns $t$ at the $n^{\text{th}}$ step} \\
		x_0 & \text{ otherwise } \\
	\end{cases}
\]

\noindent The function $a(n)$ has $X$ in its range and is computable by its definition. Hence, $X$ is enumerable.

\begin{thm}
	The intersection and union of enumerable sets is enumrable.
\end{thm}

\begin{rem}
	The complement of an enumerable set may not be enumerable.
\end{rem}
\end{document}
