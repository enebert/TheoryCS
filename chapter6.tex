\documentclass[10pt, letterpaper]{article}
%Graphic and Diagram Setup
\usepackage[margin=1in]{geometry}
\usepackage{graphicx}
\usepackage{tikz}
\usepackage[all]{xy}

%Inserting Graphics
%An example of the use of this command is
% \begin{minipage}{4in}
%   \Fig{filename}
% \end{minipage}
\newcommand{\Fig}[1]{\includegraphics[width=0.5\textwidth]{#1}}

%Math Symbol Setup
\usepackage{ulem}
\usepackage{amsmath}
\usepackage{amsthm}
\usepackage{amssymb}

%Math Fonts
\usepackage{amsfonts}
\usepackage{mathrsfs}

%Absolute Value and Norm Notation
\newcommand{\abs}[1]{\left \lvert #1 \right \rvert}
\newcommand{\norm}[1]{\left \lVert #1 \right \rVert}

%Kernel of a map
\renewcommand{\ker}{\operatorname{Ker}}

%Lie Algebra of a Lie Group
\newcommand{\lie}[1]{\operatorname{Lie}(#1)}

%Fields and Blackboard Letters
\newcommand{\field}[1]{\mathbb{#1}}
\newcommand{\A}{\mathbb{A}}
\renewcommand{\P}{\mathbb{P}}
\newcommand{\N}{\mathbb{N}}
\newcommand{\Z}{\mathbb{Z}}

%Theorems and Definitions
\newtheorem{thm}{Theorem}
\newtheorem{lemma}{Lemma}
\newtheorem{cor}{Corollary}

\theoremstyle{remark}
\newtheorem{rem}{Remark}

\theoremstyle{definition}
\newtheorem{defn}{Definition}
\newtheorem{ex}{Example}

%Fancy Header Setup
\usepackage{fancyhdr}
\fancyhf{}
\pagestyle{fancy}
\lhead{}
\chead{\bfseries $m$-reducibility \& Properties of Enumerable Sets}
\rhead{}
\lfoot{}
\cfoot{}
\rfoot{\thepage}
\renewcommand{\headrulewidth}{0.5pt}
\renewcommand{\footrulewidth}{0.1pt}

%Double Space
\usepackage{setspace}
\linespread{1.6}

%Document Content
\begin{document}
    \subsection*{$m$-reducibility}

    To prove that a certain set $X$ is undecidable we can do this by showing that: $X$ decidable implies that any enumerable set $K$
    would be decidable.

    and we know that there are undecidable enumerable sets. One can do this by reduction. Construct a total computable function $f$ 
    such that $n \in K$ if and only if $f(n) \in X$.

    \begin{defn}
        We say that a set $A \subseteq \N_0$ is $m$-reducible to a set $B \subseteq \N_0$ if there exists a computable function 
        $f:\N_0 \rightarrow \N_0$ such that $x \in A \Leftrightarrow f(x) \in B$. One says that $f$ $m$-reduces $A$ to $B$ and 
        is denoted by $A \leq_m B$.
    \end{defn}

    \begin{thm} \leavevmode 
        \begin{enumerate}
            \item [(a)] If $A \leq_m B$ and $B$ is decidable, then $A$ is decidable.
            \item [(b)] If $A \leq_m B$ and $B$ is enumerable, then $A$ is enumerable.
            \item [(c)] Reducibility is reflexive and transitive.
            \item [(d)] If $A$ is $m$-reducible to $B$, then $A^c$ is $m$-reducible ot $B^c$.
        \end{enumerate}
    \end{thm}

    \begin{proof}
        \begin{enumerate}
            \item[(a)] Let $f:A \rightarrow B$ be the computable function that does the reduction. Since $B$ is decidable there exists
            an algorithm $M$ that decides if $x \in B$ or $x \notin B$. Let $x \in A$ be arbitrary. Then $x \in A \Leftrightarrow f(x) \in B$
            apply $M$ to $f(x)$ to decide $A$. We conclude that $A$ is decidable.
            \item[(b)] Since $B$ is enumerable, we know that the semicharacteristic function for $B$ is computable. Apply the semicharacteristic 
            function to $f(x)$ and consider 
            \[
                V(x) = \begin{cases}
                    0 &\text{ if } \operatorname{semi}(f(x)) = 0 \\
                    \text{undefined} &\text{otherwise}
                \end{cases}
            \]
            which yields a computable semicharacteristic function for $A$.
            \item[(c)] Use the composition of functions.
            \item[(d)] Suppose $f:A \rightarrow B$ is the function that does the $m$-reduction. Then $x \in A \Leftrightarrow f(x) \in B$
            so $x \notin A \Leftrightarrow f(x) \notin B$ and therefore $f$ also $m$-reduces the complements. 
        \end{enumerate}
    \end{proof}

    \begin{ex}
        Show that $A \nleq_m \N_0 \setminus A$ for an enumerable undecidable set $A$.
    \end{ex}

    \begin{proof}
        Suppose not. Let $f: A \rightarrow \N_0 \setminus A$ do an $m$-reduction for an enumerable undecidable set $A$. Then $\N_0 \setminus A$
        is nonenumerable. Since $f$ $m$-reduces it will also $m$-reduce complements $f:\N_0 \setminus A \rightarrow A$
        but $A$ enumerable implies that $\N_0 \setminus A$ is enumerable, which is a contradiction.
    \end{proof}

    \begin{rem}
        Any decidable set $A$ is reducible to any set $B$ unless $B = \varnothing$ or $B = \N_0$. Here's some setup:
        $p \in B$, $q \notin B$ and $A$ is decidable. Consider the function 
        \[
            f(x) = \begin{cases}
                p &\text{ if } x \in A \\
                q &\text{ otherwise}
            \end{cases}
        \]
        If $B$ is empty we have no $p$ to use and if $B$ is $\N_0$ there is no $q$ to choose.
    \end{rem}

    \subsection*{$m$-complete sets}

    \begin{thm}
        In the class of enumerable sets, there are sets maximal with respect to $m$-reducibility, that is, sets to which any 
        enumerable set can be $m$-reduced.
    \end{thm}

    \begin{proof}
        Let $U \subseteq \N_0 \times \N_0$ be an enumerable set of pairs which is universal for the class of enumerable sets of 
        natural numbers. Consider the set $V$ of numbers of all pairs from $U$
        \[
            V =  \{\sigma(x,y) \mid (x,y) \in U \}
        \]
        Let $T$ be an arbitrary enumerable set. There exists some $n \in \N_0$ such that $U_n = T$. Hence
        \[
            x \in T \Leftrightarrow x \in U_n \Leftrightarrow (n,x) \in U \Leftrightarrow \sigma(n,x) \in V
        \]
        and therefore the function $f: T \rightarrow V$ defined by $x \mapsto \sigma(n,x)$ reduces $T$ to $V$.
    \end{proof}

    \begin{defn}
        We say that an enumerable set that is maximal with respect to $m$-reducibility is $m$-complete in the class of enumerable sets.
    \end{defn}

    \begin{rem}
        Let $f:K \rightarrow A$ be the reduction. Since $K$ is $m$-complete for a given $T$, there is $g:T \rightarrow K$ that does 
        an $m$-reduction. Hence the composition $T \rightarrow K \rightarrow A$ given by $g \circ f$ gives a reduction of $T$ to $A$.
        The conclusion follows.
    \end{rem}

    \begin{thm}
        Let $U \subset \N_0 \times \N_0$ be a G\"{o}del universal set for the class of enumerable sets. Then its diagonal section 
        $D = \{x \mid (x,x) \in U\}$  is $m$-complete.
    \end{thm}

    \begin{proof}
        The set $D$ is enumerable. Let $K$ be an arbitrary enumerable set. The set $V = \N_0 \times K$ whose sections are empty when 
        $n \notin K$ or are $\N_0$ when $n \in K$.

        Since $U$ is a G\"{o}del set, there exists a total computable function $s$ such that $V_n = U_{s(n)}$. Note that we have 
        \[
            U_{s(n)} = \begin{cases}
                \varnothing &\text{ if } n \notin K \\
                \N_0 &\text{ if } n \in K
            \end{cases}
        \]
        Well $s(n) \in U_{s(n)}$ if and only if $s(n) \in D$ for $n \in K$. We conclude that $s$ reduces $K$ to $D$.
    \end{proof}

    \begin{rem}
        We say that a set $X$ is infinite if there exists an injection $\N_0 \rightarrow X$. This is tantamount to saying there is a proper
        subset of $X$ which has cardinality $n$ for every $n \in \N_0$. We say that $X$ is effectively infinite if there exists an algorithm 
        $M$ that returns $n$ distinct elements of $X$ for every $n \in \N_0$.
    \end{rem}

    We've seen that $X$ is not enumerable if there is no algorithm that outputs exactly the elements of $X$. One way to show this  would be to find 
    some element of $X$ that lies in no enumerable set. One would perhaps say that $X$ is effectively nonenumerable if there exists $x_0 \in X$ such
    that $x_0 \notin A$ where $A$ is any enumerable set. This seems like pie-in-the-sky. Let's go through the formality now:

    Let $W$ be a G\"{o}del universal enumerable set. We than have a numbering of enumerable sets.

    \begin{defn}
        We say that $X$ is effectively nonenumerable if there exists a total computable function $d$ such that $d(z) \in X \Delta W_z$ for all $z$,
        where $\Delta$ is the set difference.
    \end{defn}

    We now embark to show that effective nonenumerability can be characterized in terms of $m$-reducibility.

    \begin{thm}
        If $A \leq_m B$ and $A$ is effectively nonenumerable, then $B$ is also effectively nonenumerable.
    \end{thm}

    \begin{proof}
        Suppose we can find a point at which $B$ differs from any enumberable set $X$. Let $f:A \rightarrow B$ be the function that does the reduction.
        Since $X$ is enumerable its preimage $f^{-1}(X)$ is also enumerable. Hence, we can find a point $x_0 \in f^{-1}(X)$ which differs from $A$. So 
        $f(x_0)$is a point where $X$ and $B$ differ. 

        \textbf{Q:} How do we effectively find $x_0$?

        Consider the enumerable set 
        \[
            V = \{(x,y) \mid (x,f(y)) \in W\}
        \]

        where $W$ is a G\"{o}del universal set that defines a numbering. The set $V$ is enumerable as the preimage of the computable map 
        $(x,y) \mapsto (x,f(y))$ and $V_n = f^{-1}(W)$. Since $W$ is a G\"{o}del universal set, there exists some total computable function 
        $s$ such that $W_{s(n)} = V_n = f^{-1}(W_n)$ for all $n$. The function $s$ maps a $W$-number of any enumerable set into a $W$-number 
        of its preimage under $f$.
    \end{proof}

    \begin{thm}
        There exist enumerable sets with effectively nonenumerable complements.
    \end{thm}

    \begin{proof}
        Let $D = \{n \mid (n,w) \in W\}$. The sets $W_n$ and $D$ do not differ at $n \in \N_0$ so $W_n$ differs from $D^c$ at $n$. We take 
        $d = \operatorname{id}$ in the defintion of effective nonenumerability.
    \end{proof}

    \begin{thm}
        The complement of any $m$-complete enumerable set is effectively nonenumerable.
    \end{thm}

    \begin{thm}
        Let $K$ be an enumerable set and $A$ an effectively nonenumerable. Then 
        \[
            \N_0 \setminus K \leq_m A \Leftrightarrow K \leq_m \N_0 \setminus A.
        \]
    \end{thm}

    \begin{proof}
        Consider the set $V = K \times \N_0$. We've seen that its sections are either empty or conincide with $\N_0$. $W$ is a G\"{o}del set so 
        we find a total function $s$ such that 
        \begin{align*}
            W_{s(n)} &= \varnothing \text{ for } n \notin K \\
            W_{s(n)} &= \N_0 \text{ for } n \in K
        \end{align*}
        Let $d$ be the function that ensures the nonenumerability of the set $A$. Then $d(s(n)) \in A$ for $n \notin K$ and 
        $d(s(n)) \notin A$ for $n \in K$. The composition of $d$ and $s$ reduces $\N_0 \setminus K$ to $A$.
    \end{proof}

    \begin{cor}
        $A$ set is effectively nonenumerable if and only if the complement of some $m$-complete set is $m$-reducible to it.
    \end{cor}

    \begin{rem}
        Not every nonenumerable set is effectively nonenumerable.
    \end{rem}

    \begin{thm}
        Any effectively nonenumerable set contains an infinite enumerable set; that is it's not immune.
    \end{thm}

    \begin{proof}
        Suppose that $A$ is effectively nonenumerable. Then we can find a point where $A$ differs from the empty set. And then a point
        in $A$ that differs from the one point set and so on. In which case, we can algorithmically find arbitrarily many distinct 
        elements.

        \textbf{Claim: } Given a finite set specified by a list of its elements, we can obtain some numbers of this set in a G\"{o}del 
        numbering of enumerable sets.

        Let us fix a certain computable numbering of finite sets. Denote by $D_n$ the $n^{\text{th}}$ finite set in this numbering.
        Then $D_n$ is the $n^{\text{th}}$ section of the enumerable set $D = \{(n,x) \mid x \in D_n\}$. Apply the defintion of 
        G\"{o}del numbering of enumerable sets.
    \end{proof}

    \subsection*{Isomorphisms of $m$-complete sets}

    Recall that a set is said to be $m$-complete if it is maximal with respect to $m$-reducibility.

    \begin{defn}
        We say that $k$ and $l$ are $A$-equivalent if $k,l \in A$ or $k,l \notin A$.
    \end{defn}

    \begin{lemma}
        Let $A$ be an $m$-complete enuemrable set. Then it is possible to algorithmically many other natural numbers that are $A$-equivalent to $n$.
    \end{lemma}

    \begin{proof}
        \textbf{Case 1: } $n \in A$

        Let $P$ be an enumerable undecidable set. Consider an enumerable set of pairs $A \times P$. It is $m$-reducible to $A$ since $A$ 
        is $m$-complete. We can always get a natural number from a pair by using a numbering. Then there is a total computable function $f$ 
        of two natural arguments with the property 
        \[
            f(n,m) \in A \Leftrightarrow n \in A \text{ and } m \in P.
        \]
        In particular, $n$ and $f(n,m)$ are $A$-equivalent. Arranging $P$ into a computable sequence $p(0), p(1), \ldots$ and obtain new numbers that 
        are $A$-equivalent to $n$. Collect these numbers into a set $X$.

        Suppose $m \in P$. Then $f(n,m) \in X$ by construction and if $m \mapsto f(n,m) \notin X$ when $m \notin P$. Thus the function therefore $X$  is 
        undecidable and hence infinite.

        \textbf{Case 2: } $n \notin A$

        Let $P$ and $Q$ be two enumerable separable sets. Consider the enumerable set of pairs $(A \times P) \cup (\N_0 \times Q)$. Let $f$ 
        be the function reducing this set of pairs to $A$. This means that $f(n,m) \in A$ if and only if $(n \in A \text{ and } m \in P)$ or $m \in Q$.
        The numbers $f(n,m)$ are $A$-equivalent for $m \in P$ and $n \in \N_0$. Let $X = \{f(n,p(0)), f(n,p(1)), \ldots\}$.

        \textbf{Claim: } This sequence contains infinitely many many terms.

        Suppose not. By assumption $X \cap A = \varnothing$. Notice that $f(n,m) \in X$ if $m \in P$ and $f(n,m) \notin X$ if $m \in Q$. Thus the 
        preimage of $X$ under the map $m \mapsto f(n,m)$ separates $P$ from $Q$. However, this preimage is decidable. This contradicts our 
        assumption that $P$ and $Q$ cannot be separated by a decidable set.

        We run these in parallel to inevitably get the desired result.
    \end{proof}

    \begin{thm}
        Let $A$ and $B$ be $m$-complete enumerable sets. Then. there exists a computable 1-to-1 correspondence $f:\N_0 \rightarrow \N_0$ that maps $A$
        onto $B$ and $x \in A \Leftrightarrow f(x) \in B$. As stated $A$ and $B$ are two $m$-complete sets that are enumerable. We proceeed to construct
        the required 1-to-1 correspondence: At the $k^{\text{th}}$ step 
        \[
            a_1 \leftrightarrow b_1, a_2 \leftrightarrow b_2, \ldots, a_k \leftrightarrow b_k
        \]
        such that $a_i \in A \Leftrightarrow b_i \in B$ for all $i$. At even steps, we can take the smallest number that has not been included with the 
        LHS of this correspondence. Using $m$-reducibility and the lemma we can produce an infinite set of terms and we can choose the counterpart by 
        using the number that has not yet appeared in the RHS. At odd steps we can do the same thing but from right-to-left.
    \end{thm}

    \begin{rem}
        There is essentially only one $m$-complete set.
    \end{rem}
\end{document}
