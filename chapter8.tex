\documentclass[10pt, letterpaper]{article}
%Graphic and Diagram Setup
\usepackage[margin=1in]{geometry}
\usepackage{graphicx}
\usepackage{tikz}
\usepackage[all]{xy}

%Inserting Graphics
%An example of the use of this command is
% \begin{minipage}{4in}
%   \Fig{filename}
% \end{minipage}
\newcommand{\Fig}[1]{\includegraphics[width=0.5\textwidth]{#1}}

%Math Symbol Setup
\usepackage{ulem}
\usepackage{amsmath}
\usepackage{amsthm}
\usepackage{amssymb}

%Math Fonts
\usepackage{amsfonts}
\usepackage{mathrsfs}

%Absolute Value and Norm Notation
\newcommand{\abs}[1]{\left \lvert #1 \right \rvert}
\newcommand{\norm}[1]{\left \lVert #1 \right \rVert}

%Kernel of a map
\renewcommand{\ker}{\operatorname{Ker}}

%Lie Algebra of a Lie Group
\newcommand{\lie}[1]{\operatorname{Lie}(#1)}

%Fields and Blackboard Letters
\newcommand{\field}[1]{\mathbb{#1}}
\newcommand{\A}{\mathbb{A}}
\renewcommand{\P}{\mathbb{P}}
\newcommand{\N}{\mathbb{N}}
\newcommand{\Z}{\mathbb{Z}}

%Theorems and Definitions
\newtheorem{thm}{Theorem}
\newtheorem{lemma}{Lemma}
\newtheorem{cor}{Corollary}

\theoremstyle{remark}
\newtheorem{rem}{Remark}

\theoremstyle{definition}
\newtheorem{defn}{Definition}
\newtheorem{ex}{Example}

%Fancy Header Setup
\usepackage{fancyhdr}
\fancyhf{}
\pagestyle{fancy}
\lhead{}
\chead{\bfseries Arithmetical Hierarchy}
\rhead{}
\lfoot{}
\cfoot{}
\rfoot{\thepage}
\renewcommand{\headrulewidth}{0.5pt}
\renewcommand{\footrulewidth}{0.1pt}

%Double Space
\usepackage{setspace}
\linespread{1.6}

%Document Content
\begin{document}
    \subsection*{$\Sigma_n$ and $\Pi_n$}

    Recall that a set $A \subset \N_0$ is enumerable if and only if there exists a decidable set $B \subset \N_0 \times \N_0$ such
    that $A$ is the projection of $B$. Let's identify sets and predicates so that we can say that a property $A(x)$ of natural numbers 
    is enumerable if and only if it can be represented in the form 
    \[
        A(x) \Leftrightarrow \exists y B(x,y)
    \]
    where $B(x,y)$ is some decidable property.

    \textbf{Q: } What can be said about other combinations of quantifiers? \\
    \textbf{Q: } What properties are representable in the form 
    \[
        A(x) \Leftrightarrow \exists y \exists z C(x,y,z)
    \]
    where $C$ is some decidable property?

    We can replace consecutive quantifiers by using one with computable numbering of pairs:
    \[
        C''(x,\sigma(y,z)) \Leftrightarrow C(x,y,z) \text{ and } A(x) \Leftrightarrow \exists w C''(x,w).
    \]

    \textbf{Q: } What properties can be represented in the form 
    \[
        A(x) \Leftrightarrow \forall y B(x,y)
    \]
    where $B(x,y)$ is a decidable property.

    \textbf{A: } The properties with enumerable negations:
    \begin{align*}
        \neg A(x) &\Leftrightarrow \neg \forall y B(x,y) \\
                  &\Leftrightarrow \exists y (\neg B(x,y))
    \end{align*}
    We are implicitly using the fact that decidability is preserved under negation.

    We now progress to a general definition:

    \begin{defn}
        A property $A$ belongs to the class $\Sigma_n$ if it can be represented in the form 
        \[
            A(x) \Rightarrow \exists y_1 \forall y_2 \exists y_3 \cdots B(x_1, y_1, y_2, \ldots, y_n)
        \]
        with $n$ alternating quantifiers and $B$ is a decidable property.

        If the $n$ alternating quantifiers starts with $\forall$, then we obtain the definition of the class $\Pi_n$.
    \end{defn}

    \begin{thm}
        \begin{enumerate}
            \item[(a)] The class $\Sigma_n$ and $\Pi_n$ does not change if we allow groups of quantifiers of the same type instead 
            of a single quantifier.
            \item[(b)] If a predicate belongs to $\Sigma_n$ then its negation belongs to $\Pi_n$ and vice versa.
        \end{enumerate}
    \end{thm}

    \begin{thm}
        \begin{enumerate}
            \item [(a)] The intersection and union of two sets of the class $\Sigma_n$ belongs to $\Sigma_n$.
            \item [(b)] The intersection and union of two sets of the class $\Pi_n$ belong to $\Pi_n$.
        \end{enumerate}
    \end{thm}

    \begin{ex}
        \begin{align*}
            A(x) &\Leftrightarrow \exists y \forall z B(x,y,z) \\
            C(x) &\Leftrightarrow \exists u \forall v D(x,u,v) \\
        \end{align*}

        Then 
        \[
            A(x) \land C(x) \Leftrightarrow \exists y \exists u \forall z \forall v [B(x,y,z) \land D(x,u,v)]
        \]
        The property is decidable and we can compbine the quantifiers to get it to belong to $\Sigma_n$.
    \end{ex}

    \begin{thm}
        The classes $\Sigma_n$ and $\Pi_n$ are "hereditary downward" with respect to $m$-reducibility in the 
        following way:
        
        If $A \leq_m B$ and $B \in \Sigma_n$ then $A \in \Sigma_n$ or $B \in \Pi_n$ then $A \in \Pi_n$.
    \end{thm}

    \subsection*{Unviersal Sets in $\Sigma_n$ and $\Pi_n$}

    \textbf{Goal: } Show that the classes $\Sigma_n$ and $\Pi_n$ are distinct for different $n$.

    \textbf{How? } We find a universal set for $\Sigma_n$ and show that it cannot belong to $\Sigma_k$ for any $k < n$.

    \begin{thm}
        For any $n$, the class $\Sigma_n$ contains a set universal for this class. The complement of this set will be 
        universal for the class $\Pi_n$.
    \end{thm}

    \begin{thm}
        Universal $\Sigma_n$-sets do not belong to the class $\Pi_n$. Similarly universal $\Pi_n$-sets do not belong to $\Sigma_n$.
    \end{thm}

    \begin{rem}
        The hierarchy is strict.
    \end{rem}
\end{document}
